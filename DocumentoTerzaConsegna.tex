\documentclass[a4paper]{article}

\usepackage[T1]{fontenc}	
\usepackage[utf8x]{inputenc}
\usepackage[italian]{babel}

 \title{Terza consegna del corso \\di \\Commercio Elettronico}
 \author{Matteo Scarpa 845087}
  \date{\today}

\begin{document}
	\maketitle
	
	Questa è la terza esercitazione del corso e prevede di creare dei suggerimenti attraverso l'utilizzo di un database (MongoDB). 
	
	\section*{Struttura dei programmi}
	Entrambi i programmi richiedono l'uso della prima esercitazione, la quale è stata modificata per rendere i testi più puliti.  I file così ottenuti vengono quindi inseriti nel database. Successivamente viene applicata loro una mapreduce per ottenere il lexicon e calcola la frequenza delle parole nei testi.
	
Successivamente genera un utente su un elenco di documenti dato	via codice attraverso una classe user. Questo viene fatto per avere del codice più modulare.

Stessa cosa viene realizzata per il database. Viene creata una classe apposita per gestire il collegamento al database e la operazioni che avvengono in esso. Questo è stato fatto per avere una variabile globale che indica il database e permette le operazioni su di esso potendo riciclare il codice per l'esercitazione successiva.\footnote{Infatti l'uso di una classe isola il codice rendendo il codice modulare e, di conseguenza, più facile da mantenere}
			
	\section*{Confronto}
	Sono stati confrontati i tempi di esecuzione dei due script (nel caso siano già stati scaricati i file e inseriti i dati nel database).
\begin{table}[h]
\begin{tabular}{|l|l|l|ll}
\cline{1-3}
 & secondaconsegna.py & terzaconsegna.py &  &  \\ \cline{1-3}
Real & 10.933 sec & 8.869 sec&  &  \\ \cline{1-3}
User & 5.340 sec & 2.995 sec&  &  \\ \cline{1-3}
Sys & 0.431 sec & 0.221sec &  &  \\ \cline{1-3}
\end{tabular}
\end{table}

Sono stati confrontati gli output dei due script nei due casi e si è scoperto che i risultati sono differenti. Personalmente mi aspettavo una minor differenza di tempi tra la seconda esercitazione (ove uso file per archiviare i documenti ottenuti) e la terza (in cui si accede ad un database). Probabilmente questo è dato dal fatto che le funzioni di accesso al database sono ottimizzate per richiedere in rapida sequenza elenchi di file mentre le funzioni di I/O su file sono pensate per il lavoro su singoli file.

\end{document}