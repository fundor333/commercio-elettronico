\documentclass[a4paper]{article}

\usepackage[T1]{fontenc}	
\usepackage[utf8x]{inputenc}
\usepackage[italian]{babel}

 \title{Seconda del corso di Commercio Elettronico}
 \author{Matteo Scarpa 845087}
  \date{\today}

\begin{document}
	\maketitle
	
	Questa è la seconda esercitazione del corso e prevede di creare dei suggerimenti attraverso due sistemi (attraverso il calcolo della distanza con il coseno e attraverso l'estrapolazione delle parole significative e confrontando queste tra i vari documenti). 
	
	\section*{Struttura dei programmi}
	Entrambi i programmi richiedono l'uso della prima esercitazione, la quale è stata modificata per rendere i testi e i file di output più semplicemente utilizzabili nella seconda esercitazione. Infatti utilizzo il file "./out/out.txt", ove sono salvati tutti itermini dei documenti scaricati con relativa frequenza e il numero di documenti così esaminato. In caso di assenza di tale documento viene invocata una funzione della prima consegna che scarica i file aggiornati e rielabora il documento.
	
	Successivamente  vengono elaborati i documenti e vengono rielaborati nella forma consona e salvati in strutture dati appropriate.  Quindi vengono simulate prima una partenza a freddo (generazione di suggerimenti di somiglianza di un documento rispetto al gruppo di interesse) e successivamente una partenza a caldo (generazione di suggerimenti di somiglianza di un gruppo di documenti rispetto al gruppo di interesse) per entrambi i metodi e salvati in file.
	
	\section*{Confronto}
	Sono stati confrontati i tempi di esecuzione dei due script (nel caso siano già stati scaricati i file e creato il dizionario).
\begin{table}[h]
\begin{tabular}{|l|l|l|ll}
\cline{1-3}
 & secondaconsegna.py & secondaconsegnabis.py &  &  \\ \cline{1-3}
Real & 10.674 sec & 0.366 &  &  \\ \cline{1-3}
User & 6.830 sec & 0.099 &  &  \\ \cline{1-3}
Sys & 0.330 sec & 0.067 sec &  &  \\ \cline{1-3}
\end{tabular}
\end{table}

Sono stati confrontati gli output dei due script nei due casi e si è scoperto che i risultati sono differenti. Questo perchè il metodo del coseno prevede il confronto tra tutte le parole dei documenti mentre il secondo script elabora solo una parte significativa del documento. 
Questo porta prima ad un risparmio notevole del tempo nel secondo caso e in una precisione maggiore in quanto ottengo risultati che non tengono conto di "parole di disturbo".

\end{document}